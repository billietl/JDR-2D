\documentclass[oneside,12pt]{article}
\usepackage[latin1]{inputenc}
\usepackage[francais]{babel}
\usepackage[francais]{layout}
\usepackage[OT1]{fontenc}
\usepackage{listings}
\usepackage{cite}
\usepackage{textcomp}
\usepackage{graphicx}
\usepackage{mathtools}
% Reglages du document
\lstset{language=bash, frame=single, breaklines=true, basicstyle=\ttfamily, keywordstyle=\bfseries}
\setlength{\hoffset}{-18pt}
\setlength{\oddsidemargin}{0pt} % Marge gauche sur pages impaires
\setlength{\evensidemargin}{9pt} % Marge gauche sur pages paires
\setlength{\marginparwidth}{54pt} % Largeur de note dans la marge
\setlength{\textwidth}{481pt} % Largeur de la zone de texte (17cm)
\setlength{\voffset}{-18pt} % Bon pour DOS
\setlength{\marginparsep}{7pt} % S\'eparation de la marge
\setlength{\topmargin}{0pt} % Pas de marge en haut
\setlength{\headheight}{13pt} % Haut de page
\setlength{\headsep}{10pt} % Entre le haut de page et le texte
\setlength{\footskip}{27pt} % Bas de page + s\'eparation
\setlength{\textheight}{668pt} % Hauteur de la zone de texte (25cm)

\begin{document}
\usefont{OT1}{bch}{m}{n}
\title{Syst\`eme de jeu de r\^ole 2D}
\date{}
\maketitle
\vspace{475pt}
\includegraphics[scale=0.06]{./img/licence.png}
\thispagestyle{empty}
\newpage
\thispagestyle{empty}
~
\newpage
\section*{L\'egal, contributions, et autres trucs}
\subsection*{Licence}
Si vous avez r\'eussi \`a obtenir ce document gratuitement et l\'egalement, c'est gr\^ace \`a la licence Creative Commons !
Youhou !
Reprenons ce que dis le site.

\fbox{%
  \begin{minipage}{0.95\textwidth}
Vous avez le droit de :
\begin{itemize}
\item
  Partager : copier et redistribuer cette oeuvre par n'importe quel moyen, peu importe le format.
\item
  Adapter : r\'earranger, transformer, et cr\'eer \`a partir de cette oeuvre. Peu importe votre but, m\^eme commercial.
\end{itemize}

L'ayant droit ne peut pas annuler ces libert\'es tant que vous respectez les termes de la licence.
Qui sont :
\begin{itemize}
\item
  Attribution:  Vous devez donner cr\'edit aux personnes ayant contribu\'e avant vous, fournir un lien vers la licence et indiquer si des changements ont \'et\'e faits.
\item
  Propagation : Si vous r\'earrangez, transformez, et/ou cr\'eez \`a partir de cette oeuvre, vous devez distribuer votre contribution sous la m\^eme licence.
\item
  Rien de plus : Vous n'\^etes pas autoris\'e \`a appliquer des conditions l\'egales ou des mesures techniques qui restreindraient l\'egalement autrui \`a utiliser l'oeuvre dans les conditions d\'ecrites par la licence.
\end{itemize}
  \end{minipage}%
}

Vous trouverez tous les termes de la licence \`a l'adresse\\https://creativecommons.org/licenses/by-sa/4.0/legalcode

Afin de faciliter le respect de l'attribution, je demanderais aux contributeur substentiels d'ajouter leur nom (ou pseudonyme) ainsi que la date \`a laquelle ils ont contribu\'e dans la liste ci-dessous.

\subsection*{Contributeurs}
\begin{itemize}
  \item Louis Billiet, le 21 f\'evrier 2015
\end{itemize}
\thispagestyle{empty}
\clearpage
\tableofcontents
\thispagestyle{empty}
\clearpage
\section{Pr\'eambule}
\setcounter{page}{1}
Pourquoi un \`eni\`eme syst\`eme de jeu de r\^ole ?
C'est tr\`es simple : je voulais faire jouer un sc\'enario qui se d\'eroulais dans l'univers de Fullmetal Alchemist (manga de Hiromu Arakawa), mais ne trouvais pas de syst\`eme qui me convenais.
Chaque syst\`eme avait au moins un des probl\`emes suivant :
\begin{itemize}
\item
  Les diff\'erents r\'esultats obtenables sont \'equiprobables.
  C'est \`a dire, pour un jet o\`u l'on consid\`ere un d\'e \`a 10 faces et un modificateur de +5 associ\'e, il est aussi probable de faire un 6, qu'un 10 ou qu'un 15.
  Soit, on a autant de chances de juste r\'eussir que de faire un exploit o\`u d'\'echouer magistralement.
  Les jeux qui ont ce probl�me sont souvent ceux o\`u les jets ne n\'ecessitent qu'un d\'e (les syst\`emes D10 comme \textit{Prophecy}, D20 comme \textit{D\&D} et D100 comme \textit{Rolemaster}).
\item
  La feuille de personnage est beaucoup trop cryptique et/ou trop longue.
  Je n'ai pas envie de passer plus d'1h avant le d\'ebut de la partie juste pour expliquer comment lire cette feuille.
  Je vise ici \textit{Eclipse phase} dont la feuille de personnage utilise beaucoup de termes peu commun, mais aussi \textit{D\&D} et \textit{Legend of the 5 Rings} dont les feuilles de personnages font plusieurs pages.
\item
  Chaque jets de d\'es n\'ecessite un nombre non-n\'egligeable de d\'es.
  Et parfois, on choisis les d\'es que l'on veut garder, ce qui peut entra\^iner les joueurs \`a tergiverser.
  J'ai vu ce principe dans les jeux \textit{Legend of the 5 Rings}, \textit{GURPS} ou encore \textit{Tunnels and Trolls}.
  Et je n'aime pas cette id\'ee.
\end{itemize}

Je voulais un syst\`eme o\`u il est beaucoup plus probable d'obtenir un r\'esultat moyen qu'avoir un r\'esultat hors norme.
Je voulais un syst\`eme facile \`a apr\'ehender, aussi bien par les joueurs confirm\'es que par les n\'eophytes.
Je voulais un syst\`eme privil\'egiant le roleplay \`a l'\'etablissement d'une strat\'egie ne prenant en compte que les r\`egles.

Cependant, je n'ai pas con\c cu un syst\`eme adapt\'e \`a la ma\^itrise de campagnes.
Ce syst\`eme n'est pas con\c cu pour g\'erer la distribution de point d'exp\'eriences ou l'\'evolution des personnages.
Ce syst\`eme n'est con\c cu que pour faire jouer des ``one-shot''.
Libre \`a vous si vous voulez adapter les r\`egles pour les utiliser en campagne.

Pour r\'esumer, c'est dans l'optique de faire jouer des ``one-shot'' \`a des conventions que j'ai mis au point ce syst\`eme.
\clearpage
\section{La feuille de personnage}
Mis \`a part les sempiternelles informations cosm\'etiques (nom et pr\'enom, \^age, traits physiques, ...), la feuille de personnage contient 5 s\'eries d'informations utiles pour appliquer les r\`egles.
Ces informations sont :
\begin{itemize}
\item
  La profession
\item
  Les caract\'eristiques
\item
  Les avantages et d\'esavantages
\item
  Les comp\'etences
\item
  L'\'equipement
\end{itemize}
\subsection{La profession}
Bien que la profession puisse \^etre principalement cosm\'etique, elle permet n\'eanmoins de d\'efinir ce que le personnage sais et sais faire.
\'Etant donn\'e que ce syst\`eme n'est pas li\'e \`a un univers, il n'existe pas de r\`egles d\'efinissant les restrictions li\'ees \`a la profession.

C'est donc au Ma\^itre du jeu de juger au cas par cas.
Il sera \'evident que dans un univers m\'edi\'eval-fantastique, un barbare n'aie pas de comp\'etences en enchantement.
Par contre, dans un univers contemporain, savoir si un \'electronicien aura des comp\'etences en informatique sera beaucoup plus subtil.
\subsection{Les caract\'eristiques}
Les caract\'eristiques permettent de sp\'ecifier les pr\'edispositions qu'ont les personnages par rapport aux t\^aches.

Le physique intervient lorsque le personnage doit faire preuve de force ou d'endurance.

Le manuel intervient lorsque le personnage doit manipuler un objet avec pr\'ecision ou doit observer.

Le mental intervient lorsque le personnage r\'efl\'echis, mobilise ses connaissances ou doit faire preuve de volont\'e.

Le social intervient lorsque le personage interagit avec les PNJs, que ce soit de mani\`ere active (discussion, commandement, ...) ou passive (intimidation, discr\'etion, ...).

Les caract\'eristiques prennent la forme d'un modificateur qui sera appliqu\'e lors d'un jet de d\'es.
\subsection{Les avantages et d\'esavantages}
Les avantages et les d\'esavantages sont disponibles \`a la discr\'etion du ma\^itre du jeu.
Il en existe 3 types :
\begin{itemize}
\item
  Ceux qui interagissent uniquement avec la difficult\'e des jets, les augmentant ou les diminuant d'un niveau\footnote{Le principe de difficult\'e sera couvert dans le chapitre \ref{gerer difficulte}.}.
  Par exemple, un personnage ayant la \textit{tremblotte} verra ses jets associ\'es au manuel augmenter en difficult\'e.
\item
  Ceux qui interagissent uniquement avec le roleplay.
  Ce sera donc au joueur de les faire impacter dans sa mani\`ere de jouer, et au ma\^itre du jeu de faire r\'eagir les PNJ en cons\'equence.
  Par exemple, un personnage \textit{aquaphobe} refusera cat\'egoriquement de traverser un cours d'eau, o\`u de sortir lorsqu'il pleut.
\item
  Ceux qui interagissent avec l'histoire / le sc\'enario / le background.
  Ce sera uniquement au ma\^itre du jeu de prendre en compte ceux-l\`a.
  Heureusement, dans un sc\'enario ``one-shot'', g\'erer ce genre d'avantage / d\'esavantage est ais\'e.
  Par exemple, un personnage ayant une \textit{dette envers la mafia} sera importun\'e pendant la partie, et ne voudra pas que les autres joueurs d\'ecouvrent de quoi il en est.
\end{itemize}
\subsection{Les comp\'etences}
Les comp\'etences d\'efinissent \`a la fois les savoirs th\'eoriques et pratiques du personnage.
\`A chaque comp\'etence est associ\'ee le jet \`a effectuer et une caract\'eristique.
Les comp\'etences disponibles d\'ependrons beaucoup de l'univers dans lequel se tiendra le sc\'enario.

Il se peut que certaines actions soient \`a cheval sur deux comp\'etences.
Dans ces cas l\`a, les deux comp\'etences seront mises \`a contribution.
Nous y reviendrons dans le chapitre \ref{choix de la competence}.
\clearpage
\section{Cr\'eer un personnage}
\subsection{Tirage des caract\'eristiques}
Quatre jets diff\'erents sont n\'ecessaires pour d\'efinir les caract\'eristique.
Ces jets sont :
\begin{itemize}
\item
  $10 + (D6 - D4)$
\item
  $10 + (D8 - D6)$
\item
  $10 + (D8 - D6)$
\item
  $10 + (D10 - D8)$
\end{itemize}

L'attribution des r\'esultats doit \^etre fait avant les jets.
Une fois les r\'esultats tir\'es, le modificateur est calcul\'ee de la mani\`ere suivante : $ modificateur = ( r\acute{e}sultat - 10 ) / 2 $.
Ensuite, s'ajoutent les moficateurs li\'es \`a l'\^age.

\begin{tabular}{| c | c | c | c | c |}
  \hline
  & physique & manuel & mental & social \\
  \hline
  enfant & $-1$ & $+1$ & $-1$ & $+1$ \\
  \hline
  adolescent & $+1$ & $0$ & $-1$ & $0$ \\
  \hline
  adulte & $+1$ & $-1$ & $0$ & $+1$ \\
  \hline
  ancien & $-1$ & $-1$ & $+2$ & $0$ \\
  \hline
  v\'en\'erable & $-1$ & $-1$ & $+2$ & $+1$ \\
  \hline
\end{tabular}
\subsection{Choix des avatages et d\'esavantages}
Le choix des avantages et d\'esavantages disponible est \`a la discretion du ma\^itre du jeu, en fonction de l'univers, bien entendu.
\`A chacun est attribu\'e un score entre 1 et 5.
Pour un avantage, plus le score est haut, plus je joueur sera avantag\'e.
Idem pour les d\'esavantages, plus le score est haut, plus le joueur sera d\'esavantag\'e.
Une fois que les avantages et les d\'esavantages sont choisis, il faut que la somme des scores des avantages \'egale celle des d\'esavantages.
\subsection{Calcul des comp\'etences}
Chaque personnage a 20 points de comp\'etences.
``D\'ebloquer'' une comp\'etence, c'est \`a dire lui attribuer le jet $ D10 - D10 $, co\^ute 1 point.
Ensuite, il est possible d'am\'eliorer les comp\'etences d\'ej\`a d\'ebloqu\'ees, c'est \`a dire augmenter le nombre de faces du premier d\'e (12, puis 20) ou diminuer le nombre de faces du deuxi\`eme d\'e (8, puis 6, puis 4, puis 3).
Chaque am\'eliorations co\^ute la quantit\'e de points d\'ej\`a d\'epens\'es dans la comp\'etence plus 1, sauf pour le passage de 12 \`a 20 faces.
Dans ce cas, l'am\'elioration co\^ute la quantit\'e de points d\'ej\`a d\'epens\'es dans la comp\'etence plus 5.
\subsection{Attribution de l'\'equipement de base}
\'Etant donn\'e que ce syst\`eme ne d\'ecris pas de syst\`eme monnetaire, libre \`a vous de choisir ce qui vous semble raisonnable, ou utile.
De toutes fa\c cons, beaucoups de sc\'enarios ``one-shot'' commencent par une visite \`a un intendant / magasin / d\'ep\^ot / \ldots
\clearpage
\section{Gestion de la difficult\'e}
Dans ce syst\`eme, il n'existe pas de r\'eussite critique o\`u d'\'eche critique.
La raison de cette absence viens du fait qu'elle casse toute la courbe de probabilit\'e.
Ce syst\`eme se base donc enti\`erement sur un syst\`eme de niveaux de r\'eussite.
\subsection{Choix de la comp\'etence}
\label{choix de la competence}
Lorsqu'un joueur entreprends une action, il devra souvent faire un jet.
\'Etant donn\'e que la liste des comp\'etences disponibles dans l'univers d\'ependra beaucoup du ma\^itre du jeu, je lui conseille de garder une version des feuilles de personnage pour lui.
Ensuite, le ma\^itre du jeu choisira parmis la liste des comp\'etences du personnage celle qui semble la plus appropri\'ee.
Si le personnage n'a aucune comp\'etence utile pour l'action entreprise, il effectuera un jet $ D10 - D10 $ associ\'e \`a une caract\'eristique et la difficult\'e sera augment\'ee d'un niveau.

Il se peut que certaines actions soient \`a cheval sur deux comp\'etences.
Dans ces cas l\`a, nous distinguerons la comp\'etence pratique de la comp\'etence th\'eorique.
La comp\'etence pratique sera celle que le personage met en \oe uvre pour atteindre son but.
La comp\'etence th\'eorique sera celle dont les connaissances lui permettra d'assurer la r\'eussite de l'action.

Prenons un exemple pour illustrer :
Alice veut faire exploser un pain de C4 au niveau d'un mur porteur de mani\`ere \`a faire s'\'ecrouler un batiment.
Le joueur fera un jet de la comp\'etence pratique associ\'ee.
Ici, un jet d'\textit{explosif}.
Si le personnage \`a une comp\'etence th\'eorique associ\'ee (ici, \textit{architecture} ou \textit{ma\c connerie}), le ma\^itre du jeu fera le jet associ\'e et le but sera atteint si le r\'esultat est sup\'erieur ou \'egal \`a z\'ero.
Sinon, le but a une chance sur deux d'\^etre atteint.
Bien entendu, cette r\`egle peut \^etre modifi\'e \`a l'aide d'un avantage \textit{chanceux} ou un d\'esavantage \textit{malchanceux}, ou si le batiment est en sc\'enarium\ldots
\subsection{Niveaux de difficult\'e}
\label{gerer difficulte}
\`A chaque action est associ\'e un niveau de difficult\'e.
Le tableau suivant vous permettra d'estimer la difficult\'e des actions :
\\ \\
\begin{tabular}{| c | c |}
  \hline
  Niveau de Difficult\'e & Estimation qualitative \\
  \hline
  -6 & Tr\`es facile \\
  \hline
  -3 & Plut\^ot facile \\
  \hline
  0 & Requiert un peu de pratique \\
  \hline
  3 & Plut�t difficile \\
  \hline
  6 & Tr\`es difficile \\
  \hline
  9 & Presque impossible \\
  \hline
\end{tabular}
\\ \\

Ensuite, le r\'esultat du jet est compar\'e au niveau de difficult\'e.
Si le jet est sup\'erieur ou \'egal � la difficult\'e, l'action est r\'eussite.
Sinon, l'action \'echoue.
\subsection{Niveaux de r\'eussite}
Une fois le jet r\'esolu, un effet pourra s'ajouter \`a l'action en fonction de la diff\'erence en le jet et le niveau de difficult\'e, \`a raison d'un effet par tranche de 3.
Par exemple, Si Bob saute d'un rebort haut, le niveau de difficult\'e estim\'e est de 0.
Si Bob fais entre 0 et 2, Bob se receptionnera correctement.
S'il fait entre -1 et -3, Bob se receptionnera mal et perdra son �lan.
Entre -4 et -6, Bob se foullera la cheville.
Entre -7 et -9, Bob se fracturera la jambe.
En revanche, si bob fais entre 3 et 5, il sautera particuli�rement loin.
Entre 6 et 8, il saura convertir sa vitesse de chute en vitesse de course.
\subsection{Gestion des combats}
\subsubsection{Initiative}
Au d\'ebut de chaque affrontement, tous les participants au combats font un jet de $ physique + manuel + D10 - D10 $ pour d\'eterminer le score d'initiative.
Chacun leurs tour, dans l'ordre des initiatives d\'ecroissantes, chaque participant font une action.
Un participant peut d\'ecider de reporter son action afin de pouvoir r\'eagir une fois jusqu'\`a son prochain tour.
\subsubsection{Blessures}
Les blessures sont g\'er\'ees de mani\`ere qualitative.
Le ma\^itre du jeu doit annoncer aux joueurs les blessures que subissent les personnages au fur et a mesure du jeu.
Il appartient au ma\^itre du jeu de garder \`a l'esprit l'\'etat des presonnages afin de les r\'epercuter sur les niveaux de difficult\'es.
\end{document}
